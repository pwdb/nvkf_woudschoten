\documentclass[a4paper,10pt]{report}

% Template for the NVKF program

\usepackage{fontspec,xltxtra,xunicode}
\defaultfontfeatures{Mapping=tex-text}
\setsansfont[Scale=MatchLowercase]{Myriad Pro}
\renewcommand{\familydefault}{\sfdefault}
\renewcommand{\ttdefault}{\sfdefault}
\usepackage{multicol}
\usepackage[a4paper, landscape, margin=0.7cm]{geometry}
\usepackage{graphicx}
\usepackage{flushend}
\usepackage{color}
\usepackage{url}
\usepackage{amsmath}
\definecolor{Blue}{rgb}{0.3,0.3,0.9}
\definecolor{DarkBlue}{rgb}{0.0,0.0,1.0}
\usepackage{polyglossia}
%\usepackage{draftwatermark}
%\SetWatermarkText{Concept}
\setdefaultlanguage{dutch}

\parskip=0pt

% space between columns
% needs to be twice the margin in order to fold in three
\setlength{\columnsep}{1.4cm}
% width of separation line between the columns
\setlength{\columnseprule}{0.2pt}

\begin{document}
\pagestyle{empty}
\thispagestyle{empty}

\begin{multicols*}{3}

%% Put the logo left top
\noindent
\begin{tabular}{cc}
    \parbox[b]{0.4\linewidth}{%
\includegraphics[width=3cm]{nvkf_logo}
}
& 
%\hspace{-0.5cm}\parbox[b]{0.6\linewidth}{%
\parbox[b]{0.6\linewidth}{%
\begin{center}
{\bfseries Leuk, maar hoe zit het nou echt?}\\ \vspace{0.15cm} 
{\bfseries XXXe NVKF-Conferentie}\\ \vspace{0.15cm}
{8 en 9 april 2016}\\ \vspace{0.15cm}
{Woudschoten Conferentiecentrum}\\
{Woudenbergseweg 54, Zeist}

\end{center}
}
\end{tabular}

%% End of the logo

% To make the header for a plenair item
\newcommand{\plenairheader}[2]{%
\begin{center}
{\color{DarkBlue}{\textbf{#1}}}\\
{\textbf{#2}}
\end{center}
}

% To make the header for a parallel session 
\newcommand{\parallelheader}[2]{%
\parskip=0pt
\noindent{{\color{Blue}{\textbf{\textit{{#1}}}}}\hfill{\textit{{#2}}}}\vspace{-0.2cm}
}

\newcommand{\parallelitem}[1]{%
\noindent \hfill{\color{DarkBlue}{\textbf{\textsc{{#1}}}}}\hfill\strut
}

\newenvironment{packed_enum}{%
    \begin{enumerate}
    \setlength{\itemsep}{1pt}
    \setlength{\itemindent}{0pt}
    \setlength{\parskip}{0pt}
    \setlength{\parsep}{0pt}
    \setlength{\leftmargin}{0pt}
}{\end{enumerate}}

\plenairheader{Plenair Programma}{Vrijdag 8 april 2016}

\begin{packed_enum}
    \item[09:30] Inschrijving en koffie
        \vfill
    \item[\textbf{10:00}] \textbf{Opening}\\\textit{Lieke Poot, voorzitter NVKF}
        \vfill
    \item[10:05] Vestibulair implantaat, Raymond van de Berg
        \vfill
    \item[10:50] Koffiepauze
        \vfill
    \item[{\color{Blue}{\textbf{11:20}}}] {\color{Blue}{\textbf{Bijeenkomsten van de kringen (1)}}}
        \vfill
    \item[12:45] Lunch\\
        {\small Bijeenkomst klinisch fysici in opleiding \hfill\textit{(Plenaire zaal)}}\\
        {\small Bijeenkomst Nederlandse Vereniging voor Klinische Informatica
        in oprichting \hfill\textit{(Zaal 25)}\strut}
        \vfill
    \item[{\color{Blue}{\textbf{14:00}}}] {\color{Blue}{\textbf{Bijeenkomsten van de kringen (2)}}}
        \vfill
    \item[15:30] Theepauze en postersessie 
        \vfill
    \item[\textbf{16:30}] {\textbf{Jonge Onderzoekersprijs (JOP)}}\\\textit{Voorzitter: Maurice Janssen}
    \item[16:30] 1
    \item[16:50] 2
    \item[17:10] 3
        \vfill
    \item[17:30] Presentatie nieuwe Klifio/KF/KFprof\\Opleidingsinitiatief van het jaar
        \vfill
    \item[\textbf{17:50}] {\textbf{Borreltijd}}
        \vfill
    \item[18:30] Diner
        \vfill
    \item[\textbf{20:30}] {\textbf{Avondspreker}}\\Geheim
\end{packed_enum}

%\columnbreak

\plenairheader{Plenair Programma}{Zaterdag 9 april 2016}

\begin{packed_enum}

\item[{\color{Blue} {\textbf{09:00}}}]{\color{Blue}{\textbf{Bijeenkomsten van de kringen (3)}}}
        \vfill
\item[10:30] Koffiepauze
        \vfill
\item[\textbf{11:00}] {\textbf{Plenaire sessie}}\\\textit{Voorzitter: Maurice Janssen}
\item[11:00] Spreker 1
\item[11:30] Spreker 2
\item[12:00] Spreker 3
        \vfill
\item[\textbf{12:30}] \textbf{Sluiting}
\item[12:40] Lunch

\end{packed_enum}

\hrule \vspace{3mm}
\parallelitem{Parallelsessies (1) vrijdagochtend 8 april}\\
\hrule
\vfill

\parallelheader{Zintuigfysica}{Zaal 24}
\begin{packed_enum}
\item[\textbf{11:20}] {\textbf{TBA}}\\\textit{Voorzitters: TBA}
%\item[11:20] Onderzoekslijnen in het UMC Utrecht\\\textit{Bert van Zanten}
%\item[11:50] Onderzoekslijnen in het LUMC\\\textit{Jeroen Briaire}
%\item[12:20] Onderzoekslijnen in het Erasmus MC\\\textit{Andre Goedegebure}
\end{packed_enum}

%\vfill
\parallelheader{Algemene Klinische Fysica \& Beeldvormende Technieken}{Kapelzaal}
\begin{packed_enum}
\item[\textbf{11:20}]\textbf{Tumor lokalisatie met I-125}\\\textit{Voorzitter: Marlies Overvelde}
%\item[11:20] Duo-presentatie Storz/St\"opler en Olympus: 3D endo\-sco\-pie-ontwikkelingen
%\item[11:50] Stand van zaken Stille Alarmeringen\\\textit{Carola van Pul}
%\item[12:00] Orthopedische koeling\\\textit{Ad Donkerlo}
%\item[12:10] Internet of Things\\\textit{Nick van Delft, KPN}
%\item[12:25] Thermal Imaging: (re)discovering the potentials for diag\-nostics and treatment evaluation in the clinical procedures\\\textit{Ruud Verdaasdonk}
\end{packed_enum}
%\columnbreak

%\parallelheader{Beeldvormende Technieken}{Atrium (A7/A8)}
%\begin{packed_enum}
%\item[\textbf{11:20}]\textbf{TBA}\\\textit{Voorzitter: TBA}
%\item[11:20] Introductie, basisprincipe van beeldreconstructiemethoden\\\textit{Antoon Willemsen, UMCG}
%\item[11:45] CT beeldreconstructiemethoden\\\textit{Koos Geleijns, LUMC}
%\item[12:05] SPECT beeldreconstructiemethoden\\\textit{Hugo de Jong, UMCU}
%\item[12:25] PET beeldreconstructiemethoden\\\textit{Johan Nuyts, KUL, Leuven}
%\end{packed_enum}

\parallelheader{Klinische Radiotherapie}{Plenaire zaal}
\begin{packed_enum}
\item[\textbf{11:20}]{\textbf{RT Sessie (1)}}\\\textit{Voorzitter: TBA}
%\item[11:20] Hypofractionation\\\textit{Jean Philippe Pignol}
%\item[11:50] MRI Linac\\\textit{Linda Kerkmeijer}
%\item[12:20] Proton Therapy\\\textit{Joachim Widder}
\end{packed_enum}

\parallelheader{Klinische Informatica}{Zaal 25}
\begin{packed_enum}
\item[\textbf{11:20}] \textbf{Alarmeringen}\\\textit{Voorzitter: TBA}
%\item[11:20] Digitaal Kernenergiewetdossier succesvol in gebruik\\\textit{Chris Peters, JBZ}
%\item[11:50] Update WAD software: kwaliteitsmetingen radiologische apparatuur\\\textit{Joost Kuijer, Vumc}
%\item[12:10] Nut en noodzaak voor een dosisregistratiesystemen (inleiding, \textit{Arjen Becht, Gelre Ziekenhuis}):\\
    %\vspace{-0.6cm} % bah, vies
    %\begin{packed_enum}
    %\item[1] Gebruikservaringen commercieel dosisregistratie software pakket \\\textit{Cécile Jeukens, MUMC}
    %\item[2] Gebruikservaringen WAD sw dosisregistratie software module \\\textit{Ralph Berendsen, Atrium}
    %\item[3] Technische implementatie aspecten en demo REM module in WAD sw  \\\textit{Anne Talsma, MZH}
    %\end{packed_enum}
\end{packed_enum}

%\columnbreak

%\vspace{-2mm}
%\hrule \vspace{0mm}
\parallelitem{Parallelsessies (2) vrijdagmiddag 8 april}
\hrule\strut
%\vfill

\parallelheader{Algemene Klinische Fysica \& Zintuigfysica}{Kapelzaal}
\begin{packed_enum}
\item[\textbf{14:00}] \textbf{Over zintuigfysica en kleurregistraties en de controles hiervan}\\\textit{Voorzitter: Homme-Auke Kooistra}
\end{packed_enum}
%\noindent Showcase cauterisatie en andere innovatieve snij-, coagulatie- en
%ablatietechnieken. De deelnemers hebben ieder 15 minuten spreektijd. 
%\begin{packed_enum}
%\item[1] HybridKnife voor gastro-enterologische/urologische procedures\\\textit{Paul Meijers, Erbe}
%\item[2] Thunderbeat: (advanced) bipolair als ultrasone techniek\\\textit{Manocher Gholghesaei, Olympus}
%\item[3] Plasmajet en AirSeal\\\textit{Don Entjes, Medical Dynamics}
%\item[4] Vessel Sealing (Marseal) en bi-polair pincet met irrigatie voor EHT\\\textit{Jos Walvius, KLS Martin}
%\item[5] Visie op de toekomst van de Energy Devices\\\textit{Jelle van Riezen, Covidien}
%\end{packed_enum}

%\vfill 

\parallelheader{Beeldvormende Technieken}{Atrium (A7/A8)}
\begin{packed_enum}
\item[\textbf{14:00}]\textbf{BVT Abstract sessie}\\\textit{Voorzitter: TBA}
%\item[14:00] Evaluatie RIVM dosisenquête: terug- en vooruitblik\\\textit{Arjen Becht, Gelre Ziekenhuis}
%\item[14:10] Toelichting op RIVM analyse CT dosisgegevens: wat kun je er mee?\\\textit{Arnold Schilham, UMCU}
%\item[14:30] Development and validation of a patient-tailored dose regime in coronary CT angiography \\\textit{Jorn van Dalen}
%\item[14:42] Imaging performance of Ingenuity TF PET/MR versus Gemini TF PET/CT\\\textit{Maqsood Yaqub}
%\item[14:54] De productie van 99mTc met een cyclotron\\\textit{Noortje de Groot}
%\item[15:06] Quantitative SPECT/CT: where are we, where are we going to?\\\textit{Eric Visser}
%\item[15:18] Radiation dose and image quality of novel mammography techniques\\\textit{Leonie Paulis}
\end{packed_enum}
 
%\vfill

%\parallelheader{Zintuigfysica}{Zaal 24}
%\begin{packed_enum}
%\item[\textbf{14:00}] \textbf{Zintuigfysica Sessie (1)}\\\textit{Voorzitters: TBA}
%\item[14:00] Onderzoekslijnen in het UMC Maastricht\\\textit{Erwin George} 
%\item[14:30] Inbedding UAC VUmc, speerpunten/research, samenwerking op campus en (inter)nationaal\\\textit{Theo Goverts}
%\item[15:00] Researchlijn op het gebied van spraakverstaan (van Plomp tot nu)\\\textit{Cas Smits}
%\end{packed_enum}

%\vfill

\parallelheader{Klinische Radiotherapie}{Plenaire zaal}
\begin{packed_enum}
\item[\textbf{14:00}] \textbf{RT Sessie (2)}\\\textit{Voorzitter: TBA}
%\item[14:00] Adaptive IGRT\\\textit{Jan Jacob Sonke}
%\item[14:30] Automatic Treatment Planning\\\textit{Sebastiaan Breedveld}
%\item[15:00] On-line Treatment Planning\\\textit{Bas Raaymakers}
\end{packed_enum}

%\vfill
\parallelheader{Klinische Informatica}{Zaal 25}
\begin{packed_enum}
\item[\textbf{14:00}] \textbf{Big data in de zorg}\\\textit{Voorzitter: TBA}
%\item[14:00] Uitleg over referentie domeinen model I-ziekenhuis? / hoe kan dit helpen bij het creëren van overzicht?\\\textit{Hans Boon, Elisabeth ziekenhuis}
%\item[14:30] Werken onder Enterprise Architectuur – ervaringen uit de praktijk en wat brengt het ons (niet) / eventueel ook andere sprekers \\\textit{Rob Malschaert, Canisius Wilhelmina Ziekenhuis} 
%\item[15:00] Regionale samenwerking op basis van Best Practices: garantie tot succes? \\\textit{Laurens de Groot, Radboud UMC} 
\end{packed_enum}

%\vfill
\hrule\vspace{3mm}
\parallelitem{Parallelsessies (3) zaterdagochtend 9 april}\\
\hrule
%\vfill

\parallelheader{Zintuigfysica}{Kapelzaal}
\begin{packed_enum}
\item[\textbf{09:00}] \textbf{Vergadering KKau}\\\textit{Voorzitters: TBA}
%\item[09:00] Lopende zaken
%\item[09:30] Hoofdbehandelaarschap en face-to-face contacten 
\end{packed_enum}
%\columnbreak

\parallelheader{Beeldvormende Technieken}{Plenair}
\begin{packed_enum}
\item[\textbf{09:00}] \textbf{WAD Protocollen}\\\textit{Voorzitters: TBA}
%\item[09:00] MRI for treatment planning\\\textit{Mariëlle Philippens}
%\item[09:30] MRI for tumour characterization\\\textit{Uulke van der Heide}
%\item[10:00] PET in motion\\\textit{Jeroen van de Kamer}
\end{packed_enum}

\parallelheader{Klinische Radiotherapie Technieken}{Plenair}
\begin{packed_enum}
\item[\textbf{09:00}] \textbf{RT Sessie (3)}\\\textit{Voorzitters: TBA}
%\item[09:00] MRI for treatment planning\\\textit{Mariëlle Philippens}
%\item[09:30] MRI for tumour characterization\\\textit{Uulke van der Heide}
%\item[10:00] PET in motion\\\textit{Jeroen van de Kamer}
\end{packed_enum}

%\vfill
\parallelheader{Algemene Klinische Fysica}{Atrium (A7/A8)}
\begin{packed_enum}
\item[\textbf{09:00}] \textbf{AKF Abstract sessie}\\\textit{Voorzitter: Sander van der Meer}
\end{packed_enum}
%\textit{Sessie in het teken van het cre\"eren van het beeld van het vakgebied Algemene Klinische Fysica en haar professionele randvoorwaarden.}
%%\begin{packed_enum}
%\item[09:00] Grenzeloze samenwerking in de regioziekenhuizen: Van Slotervaart tot Stadskanaal: Standaardisatie van werkzaamheden?\\\textit{Lieke Poot}
%\item[09:20] Stand van zaken kenniscentrum en richtlijn ontwikkeling Federatie Medisch Specialisten\\\textit{Mariken Zijlmans}
%\item[09:40] Kwaliteitsnormen t\@.b\@.v\@. uitoefening klinische fysica\\\textit{Homme-Auke Kooistra} 
%\item[09:54] Presentatie Werkgroep WINT: Risicomanagement t\@.b\@.v\@. veilig toepassen medische hulpmiddelen\\\textit{Iris Blonk}
%\item[10:15] Kwaliteitsnormen t\@.b\@.v\@. scholing van fysische agentia\\\textit{Erik Gelderblom}
%\end{packed_enum}

%\vfill
%\parallelheader{Commissie Stralingshygiëne}{Zaal 24 en 25}
%\begin{packed_enum}
%\item[\textbf{09:00}] \textbf{Commissie Stralingshygiëne: vervolgsessie CBRN -- oefening opvang radioactief besmette slachtoffers op de SEH}
%\end{packed_enum}
%\noindent {\small Op 7 november 2014 vond het NVKF symposium "Stralingsincidenten en -ongevallen
%bij patiënten" plaats. Klinisch fysici hebben toen in de praktijk kunnen
%oefenen wat een stralingsdeskundige moet doen als zich radioactief besmette
%slachtoffers melden op de SEH. Ook is een cheat-sheet voor de
%stralingsdeskundige opgesteld. Op veler verzoek worden tijdens Woudschoten
%deze oefen\-sessies herhaald. Met ervaren CBRN trainers worden alle stappen van
%de cheat-sheet doorgenomen en in realistische setting getraind. De cheat-sheet
%kan ter voorbereiding hier worden gedownload via:
%\url{http://tinyurl.com/k47zcpw}. Let op: deelname is alleen mogelijk voor
%degenen die zich hebben ingeschreven via het betreffende onderdeel op het
%congres inschrijfformulier.  Opgeven voor wachtlijst via
%\url{straling@nvkf.nl}}.

\end{multicols*}

\end{document}
