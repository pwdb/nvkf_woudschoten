\documentclass[a4paper,10pt]{report}

% Template for the NVKF program

%{{{ Preamble
\usepackage{fontspec,xltxtra,xunicode}
\defaultfontfeatures{Mapping=tex-text, Ligatures=TeX}
\setsansfont[Scale=MatchLowercase]{Myriad Pro}
\setromanfont[Scale=MatchLowercase]{Myriad Pro}
\renewcommand{\familydefault}{\sfdefault}
%\renewcommand{\ttdefault}{\sfdefault}
%\renewcommand{\itdefault}{\sfdefault}
%\renewcommand{\scdefault}{\sfdefault}
\usepackage{multicol}
\usepackage[a4paper, landscape, margin=0.7cm]{geometry}
\usepackage{graphicx}
\usepackage{flushend}
\usepackage{color}
\usepackage{url}
\usepackage{amsmath}
\definecolor{Blue}{rgb}{0.3,0.3,0.9}
\definecolor{DarkBlue}{rgb}{0.0,0.0,1.0}
\usepackage{polyglossia}
%\usepackage{draftwatermark}
%\SetWatermarkText{Concept}
\setdefaultlanguage{dutch}

\parskip=0pt

% space between columns
% needs to be twice the margin in order to fold in three
\setlength{\columnsep}{1.4cm}
% width of separation line between the columns
\setlength{\columnseprule}{0.2pt}
%}}} end preamble

\begin{document}
\pagestyle{empty}
\thispagestyle{empty}

\begin{multicols*}{3}

%{{{ Logo
%% Put the logo left top
\noindent
\begin{tabular}{cc}
    \parbox[b]{0.4\linewidth}{%
\includegraphics[width=3cm]{nvkf_logo}
}
& 
%\hspace{-0.5cm}\parbox[b]{0.6\linewidth}{%
\parbox[b]{0.6\linewidth}}} End of the logo

%{{{ To make the header for a plenair item
\newcommand{\plenairheader}[2]}}

%{{{
% To make the header for a parallel session 
\newcommand{\parallelheader}[2]}}

%{{{ parallelitem
\newcommand{\parallelitem}[1]}}

%%%{{{ enditem
%\newcommand{\itemauthor}[1]{%
%\newline\strut\hfill\mbox{\textit{#1}}%
%}%}}}

\newcommand\itemauthor[1]{{%
\unskip\nobreak\hfil\penalty50
\hskip2em\hbox{}\nobreak\hfil{\small \textbf{\textit{#1}}}%
\parfillskip=0pt \finalhyphendemerits=0 \par}}

%%%{{{ enditem
%\newcommand{\itemauthor}[1]{%
%\newline\strut\hfill\mbox{\textit{#1}}%
%\strut\hfill\hphantom{lalala}\hfill\mbox{\textit{#1}}%
%}%}}}

%{{{ packed_enum
\newenvironment{packed_enum}}}

\plenairheader{Plenair Programma}{Vrijdag 8 april 2016} %{{{

\begin{packed_enum}
    \item[09:30] Inschrijving en koffie
        \vfill
    \item[\textbf{10:00}] \textbf{Opening}\itemauthor{Lieke Poot, voorzitter NVKF}
        \vfill
    \item[10:05] Vestibulair implantaat\itemauthor{Raymond van de Berg}
        \vfill
    \item[10:50] Koffiepauze
        \vfill
    \item[{\color{Blue}{\textbf{11:20}} }] {\color{Blue}{\textbf{Bijeenkomsten van de kringen (1)}} }
        \vfill
    \item[12:45] Lunch\\
        {\small Bijeenkomst klinisch fysici in opleiding \hfill\textit{(Plenaire zaal)}}\\
        %{\small Bijeenkomst Nederlandse Vereniging voor Klinische Informatica in oprichting \hfill\textit{(Zaal 25)}\strut}
        \vfill
    \item[{\color{Blue}{\textbf{14:00} }}] {\color{Blue}{\textbf{Bijeenkomsten van de kringen (2)}} }
        \vfill
    \item[15:30] Theepauze en postersessie 
        \vfill
    \item[\textbf{16:30}] {\textbf{Jonge Onderzoekersprijs (JOP)}}\\\textit{Voorzitter: Maurice Janssen}
    \item[16:30] Quality Control in Diagnostic Imaging using Images of Patient Studies\itemauthor{Chiel den Harder}
    \item[16:50] Reducting non-actionable ICU alarms\itemauthor{Kirsten Henken}
    \item[17:10] From pixel to print: clinical implementation of 3D-printing in electron beam therapy for skin cancer\itemauthor{Richard Canters}
        \vfill
    \item[17:30] Presentatie nieuwe Klifio/KF/KFprof\\Opleidingsinitiatief van het jaar
        \vfill
    \item[\textbf{17:50}] {\textbf{Borreltijd}}
        \vfill
    \item[18:30] Diner
        \vfill
    \item[\textbf{20:30}] {\textbf{Avondspreker}}\\Koen van Mensvoort
    \end{packed_enum}%}}}

\columnbreak

\plenairheader{Plenair Programma}{Zaterdag 9 april 2016} %%{{{

\begin{packed_enum}
\item[{\color{Blue} {\textbf{09:00} }}]{\color{Blue}{\textbf{Bijeenkomsten van de kringen (3)}} }
        \vfill
\item[10:30] Koffiepauze
        \vfill
\item[\textbf{11:00}] {\textbf{Plenaire sessie}}\\\textit{Voorzitter: Maurice Janssen}
\item[11:00] Use it or lose it: the electrode-neuron interface\itemauthor{Martijn Agterberg}
\item[11:30] The Magnetized Brain\itemauthor{Lotte van Nierop}
\item[12:00] Maneuverable instrumentation: extending the reach of the surgeon\itemauthor{Paul Henselmans}
        \vfill
\item[\textbf{12:30}] \textbf{Sluiting}
\item[12:40] Lunch
\end{packed_enum}%}}}

\hrule %\vspace{3mm}
\parallelitem{Parallelsessies (1) vrijdagochtend 8 april}\\
\hrule
\vspace{3mm}
\vfill

\parallelheader{Zintuigfysica}{Zaal 24} %{{{
\begin{packed_enum}
\item[\textbf{11:20}] {\textbf{TBA}}\\\textit{Voorzitters: TBA}
\item[11:20] Leuk, maar wie geeft er nou eens overzicht over audiologisch wetenschappelijk onderzoek in het UMCG?\itemauthor{Pim van Dijk}
\item[11:35] Leuk, maar hoe zit het nu echt met muziekbeleving na cochleaire implantatie (en wat kunnen we er mee)?\itemauthor{Bert Maat}
\item[11:55] Leuk, maar hoe zit het nu echt met frequentieselectiviteit van de cochlea?\itemauthor{Pim van Dijk}
\item[12:05] Overzicht van het Nijmeegse wetenschappelijk onderzoek audiologie\itemauthor{Lucas Mens}
\item[12:35] A fast, stochastic, and adaptive model of auditory nerve responses to cochlear implant stimulation\itemauthor{Margriet van Gendt}
%\item[11:20] Leuk, maar hoe zit het nou echt met cochleaire selectiviteit?\itemauthor{Pim van Dijk}
%\item[11:50] Overzicht van het Nijmeegse wetenschappelijk onderzoek audiologie\itemauthor{Lucas Mens}
%\item[12:20] A fast, stochastic, and adaptive model of auditory nerve responses to cochlear implant stimulation\itemauthor{Margriet van Gendt}
\end{packed_enum} %}}}

%\vfill

\parallelheader{Algemene Klinische Fysica \& Beeldvormende Technieken \& Com\-mis\-sie Stralings\-hygiëne}{Kapelzaal} %{{{
\begin{packed_enum}
\item[\textbf{11:20}]\textbf{Tumorlokalisatie met I-125}\\\textit{Voorzitter: Marlies Overvelde}
\item[11:20] TBA\itemauthor{Eugenie Linthorst en Manfred van der Vlies}
\item[12:15] TBA\itemauthor{Iris van Gelder}
\item[12:40] Presentatie enquêteresultaten en paneldiscussie\itemauthor{Jurgen Mourik}
\end{packed_enum} %}}}

%\vfill

\parallelheader{Klinische Radiotherapie}{Plenaire zaal} %{{{ 
\begin{packed_enum}
\item[\textbf{11:20}] \textbf{RT Sessie (1)}\\\textit{Voorzitter: TBA}
\item[11:20] Geometrical uncertainties in the intracranial SRS treatment chain: a comprehensive overview\itemauthor{Tara vd Water}
\item[] \ldots\itemauthor{\ldots}
%\item[14:00] Adaptive IGRT\\\textit{Jan Jacob Sonke}
%\item[14:30] Automatic Treatment Planning\\\textit{Sebastiaan Breedveld}
%\item[15:00] On-line Treatment Planning\\\textit{Bas Raaymakers}
\end{packed_enum} %}}}

%\vfill

%\columnbreak
\parallelheader{Klinische Informatica}{Zaal 25} %{{{
\begin{packed_enum}
\item[\textbf{11:20}] \textbf{Alarmeringen (Help! Ik heb een alarm!)}\\\textit{Voorzitter: Guido Zonneveld}
\item[11:20] Theorie: wat zegt IHE hierover\itemauthor{Egon Scheepers (Amphia)}
\item[11:50] In de paktijk: Verpleegkunding Oproep Systeem (VOS)\itemauthor{Jannis Syntychakis (Isala)}
\item[12:10] Wat kunnen we met de informatie? Analyse achteraf\itemauthor{Carola van Pul (MMC)}
\end{packed_enum} %}}}

%\columnbreak

%\vspace{-2mm}
\hrule \vspace{0mm}
\parallelitem{Parallelsessies (2) vrijdagmiddag 8 april}
\hrule\strut
\vfill

\parallelheader{Algemene Klinische Fysica \& Zintuigfysica}{Kapelzaal} %{{{
\begin{packed_enum}
\item[\textbf{14:00}] \textbf{Over zintuigfysica, kleurregistraties en de controles hiervan}\\\textit{Voorzitter: Homme-Auke Kooistra}
\item[14:00] Introductie klinische relevantie kleurweergave (met o.a. MDL-arts en medisch fotograaf Gelre zkh)
\item[14:18] Werking optisch systeem van de mens ten aanzien van kleur\itemauthor{Gerard de wit (Bartimeus)}
\item[14:36] Ontwikkelingen op het gebied van richtlijnen en kalibratie kleurendisplays\itemauthor{Tom Kimpe (Barco)}
\item[14:54] Praktische kleurinstellingen in de huidige praktijk\itemauthor{Willy Hummel (MCL)}
\item[14:12] Ontwikkelingen kleurcontrastmetingen\itemauthor{Wendy Mahn (Gelre zkh)}
\end{packed_enum} %}}}

%\vfill 

\parallelheader{Beeldvormende Technieken}{Atrium (A7/A8)} %{{{
\begin{packed_enum}
\item[\textbf{14:00}]\textbf{BVT Abstract sessie}\\\textit{Voorzitter: TBA}
\item[14:00] Akoestische output metingen voor een Verasonics ultrageluidsysteem\itemauthor{Dennis Hulsen}
\item[14:15] In-vivo validation of interpolation-based phase offset cor\-rection in MR flow quantification: a multivendor, multi-center study\itemauthor{Mark Hofman}
\item[14:30] Dose eval\-uation for digital x-ray imaging of premature neonates\itemauthor{Teun Minkels}
\item[14:45] Repeatability of radiomics features in non-small cell lung cancer FDG-PET/CT studies: impact of reconstruction and delineation\itemauthor{Floris van Velden}
\item[15:00] Integration of micro- and macro-dosimetry for calculation of radiation doses to the islets of Langerhans due to radionuclide imaging with exendin\itemauthor{Eric Visser}
\item[15:15] EARL accreditatie metingen op de GE Discovery 710 PET-CT\itemauthor{Sjoerd Rijnsdorp}
\end{packed_enum} %}}}
 
%\vfill


\parallelheader{Klinische Radiotherapie \& Beeldvormende Technieken}{Plenaire zaal} %{{{ 
\begin{packed_enum}
\item[\textbf{14:00}] \textbf{Gezamenlijke RT-BVT Sessie}\\\textit{Voorzitter: TBA}
\item[14:00] The benefits of iterative reconstruction in dual energy CT\itemauthor{Roald Schnerr}
\item[14:30] Tumor respons meten met PET: “Hoe zit het nu echt?”\itemauthor{Michiel Sinaasappel}
\item[14:45] Accuracy and precision of partial volume cor\-rection in oncology PET/CT studies\itemauthor{Matthijs Cysouw}
\item[15:00] Videobril voor verminderen claustrofobie in tun\-nel\-ap\-pa\-ra\-tuur\itemauthor{Vincent Althof}
\item[15:15] Kwaliteitscontrole van beeldvormende apparatuur voor RT en BVT: gezamenlijk naar een hoger niveau\itemauthor{Folkert Koetsveld}
\end{packed_enum} %}}}

%\vfill

\parallelheader{Klinische Informatica}{Zaal 25} %{{{
\begin{packed_enum}
\item[\textbf{14:00}] \textbf{Big data in de zorg}\\\textit{Voorzitter: Guido Zonneveld}
\item[14:00] Big Data: alles kan, maar mag ook alles?\itemauthor{Patrick Lubbers (NKI)}
\item[14:30] Big Data in de praktijk\itemauthor{Alisa Westerhof (M\&I)}
\item[15:00] De toekomst van Big Data\itemauthor{Pieter Boon (Xomnia)}
\end{packed_enum} %}}}

%\vfill
\hrule\vspace{3mm}
\parallelitem{Parallelsessies (3) zaterdagochtend 9 april}\\
\hrule
\vfill

\parallelheader{Zintuigfysica}{Kapelzaal} %{{{
\begin{packed_enum}
\item[\textbf{09:00}] \textbf{Vergadering KKau}\\\textit{Voorzitters: TBA}
%\item[09:00] Lopende zaken
%\item[09:30] Hoofdbehandelaarschap en face-to-face contacten 
\end{packed_enum} %}}}

%\columnbreak

\parallelheader{Beeldvormende Technieken}{Plenair} %{{{
\begin{packed_enum}
\item[\textbf{09:00}] \textbf{WAD 2.0}\\\textit{Voorzitters: KlaasJan Renema \& Cécile Jeukens}
\item[09:00] Designing a Radiation Safety Plan for Médecins Sans Fron\-tiers: How to meet international standards in resource-limited conditions?\itemauthor{Leonie Paulis}
\item[09:30] Updates WAD
\item[10:00] Gebruikerservaringen
\end{packed_enum} %}}}

%\vfill

\parallelheader{Klinische Radiotherapie}{Plenaire zaal} %{{{
\begin{packed_enum}
\item[\textbf{09:00}]{\textbf{Technieken}}\\\textit{Voorzitter: TBA}
\item[09:00] Inter-fraction OAR dose variation in pancreatic SBRT eval\-uated with daily contrast-enhanced in-room diagnostic CT\itemauthor{Chrysi Papalazarou}
\item[09:18] Accuracy for dose calculations on CBCT scans of lung cancer patients using a vendor-specific approach\itemauthor{Mariska de Smet}
\item[09:36] LRPM: An Algorithm for Fast and Fuzzy Automated Multi-Objective Treatment Planning\itemauthor{Rens van Haveren}
\item[10:54] Validation of freeware-based midventilation CT calculation for upper abdominal cancer patients\itemauthor{Sandra Vieira}
\item[10:12] Robustness recipe for minimax robust optimization in in\-ten\-sity-modulated proton therapy for oropharyngeal can\-cer patients\itemauthor{Sebastian van der Voort}
\end{packed_enum} %}}}

%\vfill

\parallelheader{Algemene Klinische Fysica}{Atrium (A7/A8)} % {{{
\begin{packed_enum}
\item[\textbf{09:00}] \textbf{AKF Abstract sessie}\\\textit{Voorzitter: Sander van der Meer}
\item[09:00] De rol van klinische fysica als ``Mythbusters'' in het zieken\-huis bij de toe\-passing van (nieuwe) medische appara\-tuur\itemauthor{Ruud Verdaasdonk}
\item[09:30] First clinical experience with 3D endoscopic surgery in the St. Antonius hospital\itemauthor{Bastiaan van Nierop}
\item[09:45] Kwaliteit in apparatuur thuiszorg\itemauthor{Gert Jan Engbers}
\item[10:00] Introductie van innovatieve patiënt monitor op de Stroke Unit\itemauthor{Suzanne Oliveira-Martens}
\item[10:15] \ldots\itemauthor{\ldots}
%\item[10:15] Bekwaam = bevoegd en zo doen we dat bij medische apparatuur\itemauthor{Bianca van de Veen en Nanneke Mollink} % item wordt poster
\end{packed_enum} %}}}

\end{multicols*}

\end{document}

% vim:et foldmethod=marker ts=4 sw=4 tw=0
