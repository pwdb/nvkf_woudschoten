\documentclass[a4paper,10pt]{report}

% 2011 - XXVI - 26
% 2012 - XXVII - 27
% 2013 - XXVIII - geen conferentie
% 2014 - 28
% 2015 - 29
% 2016 - 30
% 2017 - 31

% Template for the NVKF program

%{{{ Preamble
\usepackage{fontspec,xltxtra,xunicode}
\defaultfontfeatures{Mapping=tex-text, Ligatures=TeX}
\setsansfont[Scale=MatchLowercase]{Myriad Pro}
\setromanfont[Scale=MatchLowercase]{Myriad Pro}
\renewcommand{\familydefault}{\sfdefault}
%\renewcommand{\ttdefault}{\sfdefault}
%\renewcommand{\itdefault}{\sfdefault}
%\renewcommand{\scdefault}{\sfdefault}
\usepackage{microtype}
\usepackage{multicol}
\usepackage[a4paper, landscape, margin=0.7cm]{geometry}
\usepackage{graphicx}
\usepackage{flushend}
\usepackage{xcolor}
\usepackage{color}
\usepackage{url}
\usepackage{amsmath}
\usepackage{amssymb}
\definecolor{Blue}{rgb}{0.3,0.3,0.9}
\definecolor{DarkBlue}{rgb}{0.0,0.0,1.0}
\usepackage{polyglossia}
\usepackage{enumitem}
%\usepackage{etoolbox}
\usepackage{draftwatermark}
\SetWatermarkText{Concept}
\SetWatermarkScale{1.5}
\SetWatermarkColor[gray]{0.9}
%\setdefaultlanguage{dutch}
\setdefaultlanguage{english}


\parskip=0pt

% space between columns
% needs to be twice the margin in order to fold in three
\setlength{\columnsep}{1.4cm}
% width of separation line between the columns
\setlength{\columnseprule}{0.2pt}
%}}} end preamble

\begin{document}
\pagestyle{empty}
\thispagestyle{empty}

\begin{multicols*}{3}
%{{{ To make the header for a plenair item
\newcommand{\plenairheader}[2]}}

%{{{
% To make the header for a parallel session 
\newcommand{\parallelheader}[2]}}

%{{{ parallelitem
\newcommand{\parallelitem}[1]}}

%%%{{{ enditem
%\newcommand{\itemauthor}[1]{%
%\newline\strut\hfill\mbox{\textit{#1}}%
%}%}}}

%\newcommand\itemauthor[1]{{%
%\unskip\nobreak\hfil\penalty50
%\hskip2em\hbox{}\nobreak\hfil{\small \textbf{\textit{#1}}}%
%\parfillskip=0pt \finalhyphendemerits=0 \par}}

\makeatletter
\newcommand \Dotfill {\leavevmode \cleaders \hb@xt@ .33em{\hss .\hss }\hfill \kern \z@}
\makeatother

\newcommand\itemauthor[1]{{%
\unskip\nobreak\hfil\penalty50\hskip0.33em
\hbox{}\nobreak\Dotfill\hskip0.33em{\small {\textit{#1}}}%
%\nobreak\dotfill\hskip1.0em{\normalsize \textbf{\textit{#1}}}%
\parfillskip=0pt \finalhyphendemerits=0 \par }}

%%%{{{ enditem
%\newcommand{\itemauthor}[1]{%
%\newline\strut\hfill\mbox{\textit{#1}}%
%\strut\hfill\hphantom{lalala}\hfill\mbox{\textit{#1}}%
%}%}}}

%{{{ packed_enum
\SetLabelAlign{Center}{\hfil#1\hfil}
\SetLabelAlign{Left}{#1\hfil}
\SetLabelAlign{Right}{\hfil#1}
\newenvironment{packed_enum}}}

%\makeatletter
%\patchcmd{\@item}
  %{\addvspace\itemsep}
  %{\par\kern\dimexpr.7\itemsep-.7\parskip-.7\baselineskip\relax%
   %\hrulefill%
   %\par\kern\dimexpr.3\itemsep-.3\parskip-.3\baselineskip\relax}
  %{}{}%
%\makeatother

%\strut\vfill
\noindent
\begin{center}
\includegraphics[width=3cm]{nvkf_logo}\\
{\bfseries Onderscheidend Vermogen}\\ \vspace{0.10cm} 
{\bfseries XXXI-e NVKF-Conferentie}\\ \vspace{0.10cm}
{7 en 8 april 2017}\\ \vspace{0.15cm}
{Woudschoten Conferentiecentrum}\\
{Woudenbergseweg 54, Zeist}
\end{center}
%\vfill\strut
\vfill

\plenairheader{Plenair Programma}{Vrijdag 7 april 2017} %{{{

\begin{packed_enum}
    \item[09:30] Inschrijving en koffie
        \vfill
    \item[\textbf{10:00}] \textbf{Opening}\itemauthor{Lieke Poot, voorzitter NVKF}
        \vfill
    \item[10:05] Gezondheidszorg en Technologie: een onderwijskundig perspectief\itemauthor{Heleen Miedema}\hfill {\small \textit{Technische Geneeskunde, Universiteit Twente}}
        \vfill
    \item[10:50] Koffiepauze
        \vfill
    %\item[{\color{DarkBlue}{\textbf{11:20}}}]{\color{DarkBlue}{\textbf{Parallelsessies (1)}} }
    %\item[{\color{DarkBlue}{\textbf{11:20}}}]{\color{DarkBlue}{\textbf{Plenaire Sessie}} }
    \item[\textbf{11:20}] {\textbf{Plenaire sessie}}\\\textit{Voorzitter: Maurice Janssen}
        \item[11:20] Auditory evoked potentials bij mens en dier\itemauthor{Huib Versnel}
        \item[11:50] Zicht op het onderscheid van kleine vaten\itemauthor{Walter Backes}
        \item[12:20] Is ultrageluid onderscheidend?\itemauthor{Chris de Korte}
        \vfill
    \item[12:45] \textbf{Lunch} \hfill{\small \textit{Restaurant}}\\
        {\small Bijeenkomst klinisch fysici in opleiding \hfill\textit{(Plenaire zaal)}}
        %{\small Bijeenkomst Nederlandse Vereniging voor Klinische Informatica in oprichting \hfill\textit{(Zaal 25)}\strut}
        \vfill
    \item[{\color{DarkBlue}{\textbf{14:00}}}]{\color{DarkBlue}{\textbf{Parallelsessies (1)}} }
        \vfill
    \item[15:30] Theepauze en postersessie 
        \vfill
    \item[\textbf{16:30}] {\textbf{Jonge Onderzoekersprijs (JOP)}}\\\textit{Voorzitter: TBA}
    \item[16:30] Robust optimization of VMAT in head and neck patients\itemauthor{Dirk Wagenaar}
    \item[16:50] Patiënten- en medewerkersdosis bij interventieradiologie in het MUMC+\itemauthor{Leonie Paulis}
    \item[17:10] Using TLD measurements in a 3D printed phantom and patient SPECT/CT scans to evaluate semi Monte Carlo dosimetry software\itemauthor{Willem Wormgoor}
        \vfill
    \columnbreak
\plenairheader{Plenair Programma}{Vrijdag 7 april 2017 - vervolg} %{{{
    \vfill
    %\item[{\color{DarkBlue}{\textbf{11:20}}}]{\color{DarkBlue}{\textbf{Parallelsessies (1)}} }
\item[17:30] Presentatie nieuwe klifio's en klinisch fysici\vspace{2mm}\\Opleidingsinitiatief van het jaar:\vspace{2mm}\\
    %\item[\#1] Opleiden in de regio: werkterrein èn instituut overstijgend! \itemauthor{Jaap Beintema, Arno Janssen, Germaine Jongen, Teun Minkels en Melissa van de Steeg}
    %\item[\#2] Clustersamenwerking: implementatie van een kwa\-li\-teits\-cyclus \itemauthor{Anne Lisa Wolf}
    %\item[\#3] Opleidingsinitiatief ZGT: de 3 + 1 jarige opleiding \itemauthor{Willem Wormgoor}
    \noindent
    Opleiden in de regio: werkterrein èn instituut overstijgend! \itemauthor{Jaap Beintema, Arno Janssen, Germaine Jongen, Teun Minkels en Melissa van de Steeg}
    \vspace{1mm}
    Clustersamenwerking: implementatie van een kwa\-li\-teits\-cyclus \itemauthor{Anne Lisa Wolf}
    \vspace{1mm}
    Opleidingsinitiatief ZGT: de 3 + 1 jarige opleiding \itemauthor{Willem Wormgoor}
        \vfill
    \item[\textbf{18:00}] {\textbf{Borreltijd}}
        \vfill
    \item[19:00] Diner
        \vfill
    \item[\textbf{21:00}] {\textbf{Avondspreker: Paul Louis Iske}}
    \end{packed_enum}%}}}

%\hrule\vspace{2mm}
\vfill
    %\columnbreak
\plenairheader{Plenair Programma}{Zaterdag 8 april 2017} %%{{{
%\vspace{2mm}\hrule

\begin{packed_enum}
\item[{\color{DarkBlue}{\textbf{09:00}}}]{\color{DarkBlue}{\textbf{Parallelsessies (2)}} }
        \vfill
\item[10:30] Koffiepauze
        \vfill
\item[\textbf{11:00}] {\textbf{Plenaire sessie}}\\\textit{Voorzitter: Maurice Janssen}
\item[11:00] Gamma Knives\itemauthor{Anke van Mourik \& Jannie Schasfoort}
\item[11:30] Het onderscheiden vermogen van data\itemauthor{Andre Dekker}
\item[12:00] Wat we zien, en wat we er mee doen\itemauthor{Uulke van der Heide}
        \vfill
\item[\textbf{12:30}] \textbf{Sluiting}
\vfill
\item[12:40] Lunch
\end{packed_enum}%}}}

%\columnbreak


%{{{ Logo
%% Put the logo left top
%\noindent
%\begin{tabular}{cc}
    %\parbox[b]{0.4\linewidth}{%
%\includegraphics[width=3cm]{nvkf_logo}
%}
%& 
%%\hspace{-0.5cm}\parbox[b]{0.6\linewidth}{%
%\parbox[b]{0.6\linewidth}{%
%\begin{center}
%{\bfseries Leuk, maar hoe zit het nou echt?}\\ \vspace{0.15cm} 
%{\bfseries XXXe NVKF-Conferentie}\\ \vspace{0.15cm}
%{8 en 9 april 2016}\\ \vspace{0.15cm}
%{Woudschoten Conferentiecentrum}\\
%{Woudenbergseweg 54, Zeist}
%\end{center}
%}
%\end{tabular}

%}}} End of the logo

\vfill
\vspace{4cm}
%\hrule \vspace{2mm}
%Zon onder 7 april: 20:25\\
%Zon op 8 april: 06:59

\strut
\columnbreak

\hrule \vspace{2mm}
\parallelitem{Parallelsessies (1) vrijdagmiddag 7 april} %{{{ parallel vrijdagmiddag
\vspace{2mm}\hrule\strut

\parallelheader{Klinische Radiotherapie}{Plenaire zaal} %{{{ vrijdagmiddag
\begin{packed_enum}
\item[\textbf{14:00}] \textbf{RT Sessie (1)}\\\textit{Voorzitters: TBA}
\item[14:00] First clinical experience with autoplanning in Pinnacle. Treatment of tumours in the breast and pelvis\itemauthor{Eugene Damen}
\item[14:30] Comparison of multi-scenario robustness evaluationmethods; look for a ``proton'' proof alternative to the PTV\itemauthor{Erik Korevaar}
\item[15:00] Commissioning MRL using on-line Treatment Planning\itemauthor{Bram van Asselen}
\end{packed_enum} %}}}

\vfill 

\parallelheader{Beeldvormende Technieken}{Atrium (A7/A8)} %{{{ vrijdagmiddag
\begin{packed_enum}
\item[\textbf{14:00}]\textbf{BVT Sessie (1)}\\\textit{Voorzitters: KlaasJan Renema \& Ronald Boellaard}
\item[14:00] Volumetric Assessment of Wall Shear Stress in vivo based on 4D Flow MRI \itemauthor{Aart Nederveen}
\item[14:15] Maximum Rubidium-82 activity for accurate myocardial blood flow quantification using digital PET \itemauthor{Jorn van Dalen}
\item[14:30] Metal artifact reduction of CT scans to improve PET/CT \itemauthor{Eric Visser}
\item[\textbf{14:45}] \textbf{Vergadering van de kring Beeldvormende Technieken} \\ \textit{Koersbepaling kring-BVT}
%\item[14:45] Influence of different types of acquisition techniques on procedural and staff dose in image guided interventions\itemauthor{Leonie Paulis}
%\item[15:00] Optimizing staff dose in image guided interventions by comparing phantom experiments with real-life staff doses\itemauthor{Leonie Paulis}
%\item[15:15] Personalized Feedback on Staff and Patient Dose in Image Guided Interventions – a New Era in Radiation Dose Monitoring\itemauthor{Leonie Paulis}
\end{packed_enum} %}}}
 
\vfill

\parallelheader{Zintuigfysica}{Zaal 24} %{{{ vrijdagmiddag
\begin{packed_enum}
\item[\textbf{14:00}] \textbf{Zintuigfysica Sessie (1)}\\\textit{Voorzitters: TBA}
\item[14:00] In memoriam Lucien Anteunis \itemauthor{Erwin George}
\item[14:05] Visio: Wat onderscheiden we in de videologische zorg\-keten? \itemauthor{Wim van Damme \& Harald Haalboom}
\item[14:25] UMCU: Audiologisch onderzoek en follow-up bij kinderen die een ototoxische behandeling krijgen in het Prinses Maxima Centrum voor Kinderoncologie  \itemauthor{Hiske Helleman}
\item[14:45] VUmc: Spectrale resolutie, temporele resolutie en intensiteitsbeleving bij NH en SH \itemauthor{Lucas Stam}
\item[15:05] AMC: Onderscheidend vermogen: het nut van patiëntprofielen \&
Onderscheidend vermogen van de audioloog en van het AC\itemauthor{Wouter Dreschler}
\end{packed_enum} %}}}

\vfill\strut

\columnbreak

\hrule \vspace{2mm}
\parallelitem{Parallelsessies (1) vrijdagmiddag 7 april - vervolg} %{{{ parallel vrijdagmiddag
\vspace{2mm}\hrule\strut

\parallelheader{Algemene Klinische Fysica}{Kapelzaal} %{{{ vrijdagmiddag
\begin{packed_enum}
\item[\textbf{14:00}] \textbf{AKF Sessie (1)}\\\textit{Voorzitter: Sander van der Meer}
\item[14:00] Convenant 2.0 in een Universitair Medisch Netwerk\itemauthor{Erik Gelderblom}
\item[14:22] Stroomschema medische hulpmiddelen in het MMC -- CE en niet-CE\itemauthor{Carola van der Pul en Eduard Meijer}
\item[14:44] Medische technologie: knap gevaarlijk\itemauthor{Mark den Blanken}
\item[15:08] Uitdagingen voor aantoonbare bekwaamheid medische technologie\itemauthor{Job Gutteling}
\end{packed_enum} %}}}

\vfill

\parallelheader{Klinische Informatica}{Zaal 25} %{{{ vrijdagmiddag
\begin{packed_enum}
\item[\textbf{14:00}] \textbf{Klinische Informatica Sessie }\\\textit{Voorzitter: Guido Zonneveld}
\item[14:00] NVZ Versnellingsprogramma Informatie-uitwis\-seling Pa\-tiënt en Pro\-fessional \itemauthor{Maarten Fischer}
\item[14:30] De Medmij standaard \itemauthor{Irene Duijvendijk}
\item[15:00] Discussie: Uitwisseling data met de patient \itemauthor{Guido Zonneveld \& Roanda Fokkens}
\end{packed_enum} %}}}

\vfill

\parallelheader{Biomedisch Technologen in de Zorg}{Zaal 22/23} %{{{ vrijdagmiddag
\begin{packed_enum}
\item[\textbf{14:00}] \textbf{Veilig gebruik, van een veilig apparaat in een veilige omgeving}\\\textit{Voorzitters: TBA}
\item[14:00] Titel \itemauthor{Spreker}
\item[14:50] Titel \itemauthor{Spreker}
\item[15:10] Titel \itemauthor{Spreker}
\end{packed_enum} %}}}

%}}} parallel vrijdagmiddag

\vfill\vspace{5cm}\strut

%\vfill
%\strut
%\columnbreak

%\strut
\columnbreak
\hrule\vspace{2mm}
\parallelitem{Parallelsessies (2) zaterdagochtend 8 april} %{{{ Parallel zaterdagochtend
\vspace{2mm}\hrule\strut

\parallelheader{Zintuigfysica}{Zaal 24} %{{{ zaterdagochtend
\begin{packed_enum}
    %\item[\textbf{09:00}] \textbf{Vergadering KKau}\\\textit{Voorzitters: TBA}
\item[\textbf{09:00}] \textbf{Vergadering KKau}
%\item[09:00] Lopende zaken
%\item[09:30] Hoofdbehandelaarschap en face-to-face contacten 
\end{packed_enum} %}}}

\vfill

\parallelheader{Beeldvormende Technieken}{Atrium (A7/A8)} %{{{ zaterdagochtend
\begin{packed_enum}
\item[\textbf{09:00}] \textbf{BVT Sessie (2)}\\\textit{Voorzitters: Jan Habraken \& C\'ecile Jeukens}
\item[09:00] Loodcomposiet vs.\ loodvrije schorten: welke zijn beter? \itemauthor{Steffie Peters}
\item[09:15] Pressure standardized compression: a safe replacement for force standardized compression in mammography?\itemauthor{Lida Dam-Vervloet} 
\item[09:30] Mammografie: Overdrachtseigenschappen bij 2D en Tomosynthese beelden \itemauthor{Leo Romijn}
\item[09:45] Fysica: Overdrachtseigenschappen van projectie-rönt\-gen\-systemen\itemauthor{Frits van der Meer}
\item[10:00] Samenvatting Dosismonitoring bijeenkomst 9 maart\itemauthor{Peter Brinks}
\item[10:15] Bridging the gap: easy to implement QC for diagnostic image quality \itemauthor{Bastiaan van Nierop}
\end{packed_enum} %}}}

\vfill

\parallelheader{Algemene Klinische Fysica}{Kapelzaal} % {{{ zaterdagochtend
\begin{packed_enum}
\item[\textbf{09:00}] \textbf{AKF Sessie (2)}\\\textit{Voorzitter: Eduard Meijer}
\item[09:00] A Dashboard For Managing Medical Technology\itemauthor{Nikita Kruis}
\item[09:15] Duurzaam bekwaam gebruik van medische technologie\itemauthor{Marcia Emmer}
\item[09:30] Feasibility and reproducibility of local pulse wave velocity measurements using ultrafast ultrasonic imaging\itemauthor{Wouter Gevers}
\item[09:45] Landelijke inventarisatie incidentmeldingen medische apparatuur \itemauthor{Tim van der Goot}
\item[10:00] Comparison and use 3D scanners to improve the quantification of medical images (surface structures and volumes) during follow up \itemauthor{Ruud Verdaasdonk}
\item[10:15] Dosisoptimalisatie bij ingebruikname van een nieuwe CT-scanner \itemauthor{Robert Elfrink}
\end{packed_enum} %}}}

\vfill

\columnbreak

\hrule\vspace{2mm}
\parallelitem{Parallelsessies (2) zaterdagochtend 8 april - vervolg} %{{{ Parallel zaterdagochtend
\vspace{2mm}\hrule\strut
%}}} parallel zaterdagochtend
\vfill
\parallelheader{Klinische Radiotherapie}{Plenaire zaal} %{{{ zaterdagochtend
\begin{packed_enum}
\item[\textbf{09:00}]{\textbf{RT Sessie (2)}}
\item[09:00] Late toxicity in the HYPRO randomized trial analyzed by automated planning and intrinsic NTCP-modelling\itemauthor{Abdul Wahab M. Sharfo}
\item[09:22] Automated treatment planning for prospective QA in the TRENDY randomized trial on liver-SBRT for HCC\itemauthor{Steven Habraken}
\item[09:44] Locoregionale radiotherapie van het mammacarcinoom in NL: survey van gebruikte technieken en ontwikkelingen\itemauthor{Duncan den Boer}
\item[10:06] Automatische kwaliteitscontrole geometrische nauw\-keur\-igheid MRI voor toepassing in de radiotherapie\itemauthor{Danique Barten}
\end{packed_enum} %}}}


\vfill

%\columnbreak
%\vspace{-0.2cm}
\hrule\vspace{2mm}
\parallelitem{Posters} %{{{ Posters 
\vspace{2mm}\hrule\strut
\vspace{-0.5cm}
\begin{packed_enum}
% Twee posters vervallen
%\item[\#1] Eigen ontwikkelde software tool voor ziekenhuisbrede dosismonitoring \itemauthor{Loes Sauren} 
%\item[\#8] Ervaringen met een dosisregistratiesysteem\itemauthor{Janneke Ansems}
%\item[\#4] Influence of different types of acquisition techniques on procedural and staff dose in image guided interventions\itemauthor{Leonie Paulis}
%\item[\#6] Optimizing staff dose in image guided interventions by comparing phantom experiments with real-life staff doses\itemauthor{Leonie Paulis}
%\item[\#7] Personalized Feedback on Staff and Patient Dose in Image Guided Interventions – a New Era in Radiation Dose Monitoring\itemauthor{Leonie Paulis}
\item[\#1] Introduction of Near-Infrared Auto-Fluorescence Imaging to Improve Surgical Precision in the Thyroid Area\itemauthor{Marieke Lijnkamp-Pluijmert}
\item[\#2] Comparison of image quality and radiation dose between conventional 2D full field digital mammography and synthetic 2D mammography\itemauthor{Lida Dam-Vervloet}
\item[\#3] Angiografie met Clarity: halvering dosis, behoud beeldkwaliteit\itemauthor{Han Keijzers}
\item[\#4] Improved class solutions for prostate brachytherapy planning via evolutionary machine learning\itemauthor{Stef Maree}
\item[\#5] Impact of model and dose uncertainty on NTCP-model-based patient selection for proton therapy\itemauthor{Rik Bijman}
\item[\#6] Prospective validation of the usefulness of treatment planning QA for  automatic treatment planning\itemauthor{Yibing Wang}
\item[\#7] Robustness of IMRT and VMAT for interfraction motion in locoregional breast irradiation\itemauthor{Richard Canters}
\item[\#8] Does intrafraction motion between breath holds during breast cancer treatment impact on delivered dose?\itemauthor{Martijn Kusters}
\item[\#9] Reproducibility and stability of vmDIBHs during breast cancer treatment measured using a 3D camera\itemauthor{Martijn Kusters}
\item[\#10] NCS rapport 24 metingen op vier nieuwe Elekta VersaHD toestellen\itemauthor{Julienne Nijst-Brouwers}
\end{packed_enum} %}}}
%\vfill
%\strut

%\vfill
%\strut
%\columnbreak


\end{multicols*}

\end{document}

% vim:et foldmethod=marker ts=4 sw=4 tw=0
